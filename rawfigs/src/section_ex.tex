\documentclass{article}
\usepackage{tikz}
\usetikzlibrary{shapes.geometric}
\usepackage{geometry}
\geometry{paperwidth=15cm,paperheight=15cm,top=2pt,left=0pt,right=0pt,bottom=0pt}
\pagestyle{empty}
\begin{document}
\pgfdeclarelayer{background}
\pgfsetlayers{background,main}
\centering
\begin{tikzpicture}
\clip(-7,-5) rectangle (7,5);
\node[shading=ball,inner color=red,outer color=white,ellipse,minimum height=3cm,minimum width=5cm](a) at (0,0) {};
\begin{pgfonlayer}{background}
\foreach \x in {10,12,...,80}
{
\draw (a.center) .. controls +(\x:10cm)  and (0:7cm) .. (a.east);
\draw (a.center) .. controls +(90+\x:10cm)  and (180:7cm) .. (a.west);
\draw (a.center) .. controls +(-\x:10cm)  and (0:7cm) .. (a.east);
\draw (a.center) .. controls +(-90-\x:10cm)  and (180:7cm) .. (a.west);
}
\end{pgfonlayer}
\draw[color=red!20,fill](a.west) ..controls(60:1cm) and (120:1cm) .. (a.east) 
		..controls (-120:1cm) and (-60:1cm)..(a.west);
\draw(a) node(b)[shading=ball,inner color=red,outer color=red!30,circle,minimum size=1.2cm,inner sep=0pt,outer sep=0pt]{};
\draw[color=red!10,fill](b.north) ..controls(-30:2mm) and (30:2mm) .. (b.south) 
		..controls (150:2mm) and (210:2mm)..(b.north);
\draw(b) node(c)[shading=ball,inner color=red,outer color=red,circle,minimum size=3mm,inner sep=0pt,outer sep=0pt]{};
\end{tikzpicture}
\end{document}
